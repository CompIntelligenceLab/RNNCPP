\documentclass{article}
\usepackage{amsmath}
\begin{document}
\section{Derivation of back propagation}
Let us first consider a feed-forward network with a single layer. The output takes the following form: 
$$
y^k_i = \sum_{j}  f^k_i(w^{k,l}_{i,j} y^l_{j} + b^k_i)
$$
summed over the nodes of layer $l$. 
Information propagates from layer $l$ to layer $k$. The output to layer $l$, evaluated at node $j$ is $y^l_j$. 
Calculate the derivative of $y^k_i$ with respect to $w^{k,l}(p,q)$: 
\begin{align}
\frac{dy^k_i}{dw^{k,l}_{p,q}}
   &= \sum_{j} {f^k}' \delta_{i,p}\delta{j,q} y^l_j \\
   &= {f^k}' \delta_{i,p} y^l_q
\end{align}
where $f'(x)$ is evaluated at $x=w^{k,l}_{i,j} y^l_{j} + b^k_i$. 
%\end{document}

Similarly, taking the derivative with respect to $b^k_j$,the result is: 
\begin{align}
\frac{dy^k_i}{db^{k}_{p}}
   &= \sum_{j} {f^k}' \delta_{i,p} \\
\end{align}

It is best to include the biases inside the weight matrix, by adding one row to the weight matrix.
The biases will be the first row. Thus, only the formula 
$$
\frac{dy^k_i}{dw^{k,l}_{p,q}} = {f^k}' \delta_{i,p} y^l_q
$$
is relevant. in vector form, this relation becomes
$$
\frac{dy^k}{dw^{k,l}} = {f^k}' \delta_{i,p} y^l_q
$$

Let us call $z$ the argument of the activation function. In later $k$, $z$ has $n_k$ elements. 
Layer 0 is the input layer. 
Thus: 
\begin{align}
z^k_i &= \sum_j w^{k,l}_{i,j} y^l_{j} \\
z^k   &=  W^{k,l} Y^l
\end{align}
where the 2nd equation is written in vector form. The matrix $W^{k,l}$, connecting layers $l$ and $k$ contains the biases
of layer $k$ in the first row. The output of layer $k$ is thus:
\begin{align}
y^k_i &= f^k(z^k_i)
y^k = f^k(z^k)
\end{align}
By composite differentiation, one has
$$
\frac{dy^k}{dW} = \frac{dy^k}{dz^k} \frac{dz^k}{dW^{k,l}}
$$
We now consider two layers: 
In the first layer: 
\begin{align}
Z^1 &=  W^{1,0} Y^0 \\
Y^1 &= f^1(Z^1)
\end{align}
where $Y^0=X$, the input to the network. 
In the second layer, 
\begin{align}
Z^2 &=  W^{2,1} Y^1 \\
Y^2 &= f^2(Z^2)
\end{align}
and $Y^2=Y$ is the output of the network. 

Compute the derivative of $Y^2$ with respect to $W^{2,1}$ and with respect to $W^{1,0}$: 
\begin{align}
\frac{dY^2}{dW^{2,1}} &= {f^2}'(Z^2) \frac{dZ^2}{dW^{2,1}} \\
\frac{dY^2}{dW^{1,0}} &= {f^2}'(Z^2) \frac{dZ^2}{dW^{1,0}} \\
                      &= {f^2}'(Z^2) \frac{dZ^2}{dY^{1}} \frac{dY^{1}}{dW^{1,0}} \\
                      &= {f^2}'(Z^2) W^{2,1} {f^1}'(Z_1)\frac{dZ^1}{dW^{1,0}} \\     % not sure about row 0
                      &= {f^2}'(Z^2) W^{2,1} {f^1}'(Z_1) Y_0     
\end{align}

Derivative of $Y=A*X$ with respect to $A$. Let $Y$ be a vector with $n$ rows, $X$ a vector with $m$ rows, and $A$ a $n\times m$ matrix. 
We have the matrix-vector product, 
\begin{align}
Y_i = \sum_j A_{ij} X_j
\end{align}
Taking the derivative of $Y_i$ with respect to $A_{kl}$, one gets: 
\begin{align}
\frac{dY_i}{dA_{kl}} &= \sum_j \delta_{i,k}\delta{j,l} X_j \\
                     &=   \delta_{i,k} X_l \\
                     &=   I \otimes X 
\end{align}
where $I$ is the identity matrix of dimension $n\times n$. $I\otimes X$ is a tensor product. 

In general, the following relationship holds between two vectors $a$, $b$, and a matrix $W$: 
$$
\frac{da^TWb}{dW} = a b^T
$$

\section{Nathan's Notes}

\subsection{Cost}
For now I am just quickly putting the formulas in here. I will explain and document more in the future once these things are implemented and working.

More thought needs to be put into this, but for getting things implemented quickly I will use the RMS cost function.

Cost Function
\begin{align}
J(W) = \frac{1}{2m} \bigg [\sum_{i=1}^m \sum_{k=1}^K (h_W(x_k^{(i)}) -y_k^{(i)})^2 \bigg ]
\end{align}

Where $m$ is the batch size, $K$ is the number of elements in each sample, $W$ is the matrix of weights, and $h_W$ is the hypothesis or output of the network. The input is denoted by $x$, and the its corresponding label is $y$. \\

For different cost functions, here is a list I found online - http://stats.stackexchange.com/questions/154879/a-list-of-cost-functions-used-in-neural-networks-alongside-applications

\subsection{Back Propagation}


\end{document}

